\documentclass{article}

\usepackage{hyperref}
\usepackage{appendix}

\begin{document}

\title{HGIS - GIS in Haskell}
\author{Vanessa McHale}

\maketitle

\begin{abstract}
This document contains explanations of some of the math behind hgis works. It
also serves as documentation supplemental to haddock, presernting several
examples and sample outputs. 
\end{abstract}

\tableofcontents

\section{Parsing a shapefile}

\section{Geometry on a sphere}

Computing perimeters on a sphere is relatively straightforward: 

\subsection{Measuring compactness}

\section{Making maps}

\subsection{Projections}

\begin{appendices}
  \section{Installation}
  HGIS is supported and tested with stack, available from
  \url{https://haskellstack.org}. 

  \section{Using the libaries}
\end{appendices}

\end{document}
